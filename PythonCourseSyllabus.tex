\documentclass[11pt, a4paper]{article}
%\usepackage{geometry}
\usepackage[inner=1.5cm,outer=1.5cm,top=2.5cm,bottom=2.5cm]{geometry}
\pagestyle{empty}
\usepackage{graphicx}
\usepackage{fancyhdr, lastpage, bbding, pmboxdraw}
\usepackage[usenames,dvipsnames]{color}
\definecolor{darkblue}{rgb}{0,0,.6}
\definecolor{darkred}{rgb}{.7,0,0}
\definecolor{darkgreen}{rgb}{0,.6,0}
\definecolor{red}{rgb}{.98,0,0}
\usepackage[colorlinks,pagebackref,pdfusetitle,urlcolor=darkblue,citecolor=darkblue,linkcolor=darkred,bookmarksnumbered,plainpages=false]{hyperref}
\renewcommand{\thefootnote}{\fnsymbol{footnote}}

\pagestyle{fancyplain}
\fancyhf{}
\lhead{ \fancyplain{}{Modeling \&  Evaluation} }
%\chead{ \fancyplain{}{} }
\rhead{ \fancyplain{}{September 26, 2014} }
%\rfoot{\fancyplain{}{page \thepage\ of \pageref{LastPage}}}
\fancyfoot[RO, LE] {page \thepage\ of \pageref{LastPage} }
\thispagestyle{plain}

%%%%%%%%%%%% LISTING %%%
\usepackage{listings}
\usepackage{caption}
\DeclareCaptionFont{white}{\color{white}}
\DeclareCaptionFormat{listing}{\colorbox{gray}{\parbox{\textwidth}{#1#2#3}}}
\captionsetup[lstlisting]{format=listing,labelfont=white,textfont=white}
\usepackage{verbatim} % used to display code
\usepackage{fancyvrb}
\usepackage{acronym}
\usepackage{amsthm}
\VerbatimFootnotes % Required, otherwise verbatim does not work in footnotes!



\definecolor{OliveGreen}{cmyk}{0.64,0,0.95,0.40}
\definecolor{CadetBlue}{cmyk}{0.62,0.57,0.23,0}
\definecolor{lightlightgray}{gray}{0.93}



\lstset{
%language=bash,                          % Code langugage
basicstyle=\ttfamily,                   % Code font, Examples: \footnotesize, \ttfamily
keywordstyle=\color{OliveGreen},        % Keywords font ('*' = uppercase)
commentstyle=\color{gray},              % Comments font
numbers=left,                           % Line nums position
numberstyle=\tiny,                      % Line-numbers fonts
stepnumber=1,                           % Step between two line-numbers
numbersep=5pt,                          % How far are line-numbers from code
backgroundcolor=\color{lightlightgray}, % Choose background color
frame=none,                             % A frame around the code
tabsize=2,                              % Default tab size
captionpos=t,                           % Caption-position = bottom
breaklines=true,                        % Automatic line breaking?
breakatwhitespace=false,                % Automatic breaks only at whitespace?
showspaces=false,                       % Dont make spaces visible
showtabs=false,                         % Dont make tabls visible
columns=flexible,                       % Column format
morekeywords={__global__, __device__},  % CUDA specific keywords
}

%%%%%%%%%%%%%%%%%%%%%%%%%%%%%%%%%%%%
\begin{document}
\begin{center}
{\Large \textsc{Introduction to Python Programming}}
\end{center}
\begin{center}
Spring 2017
\end{center}
%\date{January 19, 2017}

\begin{center}
\rule{6in}{0.4pt}
\begin{minipage}[t]{.75\textwidth}
\begin{tabular}{llcccll}
\textbf{Instructor:} & Enes Kemal Ergin & & &  & \textbf{Time:} & T-T 14:30 -- 16:00 \\
\textbf{Email:} &  \href{mailto:eergin@na.edu}{eergin@na.edu} & & & & \textbf{Place:} & 8th Floor Research Room
\end{tabular}
\end{minipage}
\rule{6in}{0.4pt}
\end{center}
\vspace{.5cm}
\setlength{\unitlength}{1in}
\renewcommand{\arraystretch}{2}

\noindent\textbf{Course Page:} \begin{enumerate}
\item \url{https://github.com/NAU-ACM/IntroductionToPython}
\end{enumerate}

\vskip.15in
\noindent\textbf{Office Hours:} Every day, 16:00 - 17:30 at Tutoring Center


\vskip.15in
\noindent\textbf{Objectives:}  The purpose of this course is showing how to program using Python Language. Throughout this 8-week course we will learn version control system's logic, how to use Git and GitHub to learn how to be efficient developer, Python's built-in data types, Python's syntax, control structures, functions, modules, and classes. During the course period we will complete so many hands-on exercises together and weekly challenges. After completing this introductory Python course, you will be ready to go to next step to learn deeper concepts in Python Programming and it's packages, such as; Django or Flask for web development, pandas for Data science, Matplotlib for plotting, pygame for game development, scipy for scientific applications, numpy for huge numerical applications, and more.


At the end of the course, a successful student should be able to:
\begin{itemize}
\item understand the syntax and written Python code,
\item comprehend and implement the most fundamental algorithms with Python,
\item develop programs from basic to complex,
\end{itemize}


\vskip.15in
\noindent\textbf{Prerequisites:}
\begin{itemize}
\item Computer with Anaconda Python Distribution installed 
\item GitHub Account
\item Internet Connection
\item Enthusiasm
\end{itemize}


\vspace*{.15in}

\noindent \textbf{Tentative Course Outline:}
\begin{center} 
\begin{minipage}{5in}
\begin{flushleft}
%Chapter 1 \dotfill ~$\approx$ 3 days \\
{\color{darkgreen}{\Rectangle}} ~About Python and Setting Up the Environment		\\
{\color{darkgreen}{\Rectangle}} ~Data and Expressions	\\
{\color{darkgreen}{\Rectangle}} ~Control Structures \\
{\color{darkgreen}{\Rectangle}} ~Sequences		\\
{\color{darkgreen}{\Rectangle}} ~Functions		\\
{\color{darkgreen}{\Rectangle}} ~Modular Design \\
{\color{darkgreen}{\Rectangle}} ~Writing and Reading Files	\\
{\color{darkgreen}{\Rectangle}} ~Dictionaries and Sets		
\end{flushleft}
\end{minipage}
\end{center}

%\vspace*{.15in}
%\noindent\textbf{Grading Policy:} Homework and quizzes (20\%),  Midterm 1 (25\%), Midterm 2 (25\%), Final (30\%). %Four Projects (40\% = 4 * 10\%)

%\vskip.15in
%\noindent\textbf{Important Dates:}
%\begin{center} \begin{minipage}{3.8in}
%\begin{flushleft}
%Midterm \#1      \dotfill ~\={A}b\={a}n 16, 1393 $\equiv$ November 7, 2014 \\
%Midterm \#2      \dotfill ~\={A}zar 21, 1393 $\equiv$ December 12, 2014 \\
%Project Deadline \dotfill ~Month Day \\
%Final Exam       \dotfill ~Dey 18, 1393 $\equiv$ January 8, 2015 \\
%\end{flushleft}
%\end{minipage}
%\end{center}




%%%%%% END 
\end{document} 